\documentclass[a4paper,11pt]{article}
\usepackage[utf8]{inputenc}
\usepackage[catalan]{babel}
\usepackage{amsmath}
\usepackage{amsthm}
\usepackage{amssymb}
\usepackage{graphicx}
\usepackage[left=3cm,top=3cm,right=2.5cm,bottom=2.5cm]{geometry}
\usepackage{verbatim}
\usepackage{hyperref}

\title{Especificació dels jocs de prova per HeteSim}
\author{Grup 14.2}
\date{ }

\begin{document}
\maketitle
\tableofcontents
\newpage

El graf utilitzat per les dues proves es pot carregar usant l'entrada de
\verb hetesimgraf.txt .

\section{jpWithoutInter}
 \subsubsection*{Descripció}
 Els casos d'ús que es proven són:
 \begin{itemize}
  \item Calcular el HeteSim de dos nodes (cas sense tipus intermig).
 \end{itemize}
 Es carrega un graf i es calcula el HeteSim entre un autor i un terme 
 mitjançant el camí $APAPT$. La sortida s'ha verificat a mà.
 
 \subsubsection*{Objectius}
 Calcular correctament el HeteSim en el cas d'un camí de mida parella (en nombre
 de relacions).
 
 \subsubsection*{Entrada}
 Al fitxer \verb jpWithoutInter.in .
 
 \subsubsection*{Sortida}
 Al fitxer \verb jpWithoutInter.out .
 
 \subsubsection*{Resultat}
 Correcte.
 \newpage
 
 
\section{jpWithInter}
 \subsubsection*{Descripció}
 Els casos d'ús que es proven són:
 \begin{itemize}
  \item Calcular el HeteSim de dos nodes (cas amb tipus intermig).
 \end{itemize}
 Es carrega un graf i es calcula el HeteSim entre un autor i un paper 
 mitjançant el camí $APTPAP$. La sortida s'ha verificat a mà.
 
 \subsubsection*{Objectius}
 Calcular correctament el HeteSim en el cas d'un camí de mida senar (en nombre
 de relacions).
 
 \subsubsection*{Entrada}
 Al fitxer \verb jpWithInter.in .
 
 \subsubsection*{Sortida}
 Al fitxer \verb jpWithInter.out .
 
 \subsubsection*{Resultat}
 Correcte.
 \end{document}