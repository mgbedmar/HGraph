\documentclass[a4paper,11pt]{article}
\usepackage[utf8]{inputenc}
\usepackage[catalan]{babel}
\usepackage{amsmath}
\usepackage{amsthm}
\usepackage{amssymb}
\usepackage{graphicx}
\usepackage[left=3cm,top=3cm,right=2.5cm,bottom=2.5cm]{geometry}
\usepackage{verbatim}
\usepackage{hyperref}

\title{Especificació dels jocs de prova per Graph}
\author{Grup 14.2}
\date{ }

\begin{document}
\maketitle
\tableofcontents
\newpage

\section{jpAddRemRem}
 \subsubsection*{Descripció}
 Els casos d'ús que es proven són:
 \begin{itemize}
  \item Afegir un node al graf.
  \item Esborrar un node del graf.
  \item Esborrar un node del graf (cas en que no existeix).
 \end{itemize}
 S'afegeix un autor i s'esborra dues vegades. A la segona s'avisa que no existeix cap node.
 
 \subsubsection*{Objectius}
 Afegir i esborrar nodes correctament, avisant a l'usuari quan no existeix el node introduït.
 
 \subsubsection*{Entrada}
 Al fitxer \verb jpAddRemRem.in .
 
 \subsubsection*{Sortida}
 Al fitxer \verb jpAddRemRem.out .
 
 \subsubsection*{Resultat}
 Correcte.
 \newpage
 