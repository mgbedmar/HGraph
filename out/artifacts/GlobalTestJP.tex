\documentclass[a4paper,11pt]{article}
\usepackage[utf8]{inputenc}
\usepackage[catalan]{babel}
\usepackage{amsmath}
\usepackage{amsthm}
\usepackage{amssymb}
\usepackage{graphicx}
\usepackage[left=3cm,top=3cm,right=2.5cm,bottom=2.5cm]{geometry}
\usepackage{verbatim}
\usepackage{hyperref}

\title{Especificació dels jocs de prova pel sistema global}
\author{Grup 14.2}
\date{ }

\begin{document}
\maketitle
\tableofcontents
\newpage

\section{addNode}
 \subsubsection*{Descripció}
 Els casos d'ús que es proven són:
 \begin{itemize}
  \item Afegir un node al graf.
 \end{itemize}
 S'afegeix un node.
 
 \subsubsection*{Objectius}
 Afegir un node al graf.
 
 \subsubsection*{Entrada}
 Al fitxer \verb addNode.in .
 
 \subsubsection*{Sortida}
 Al fitxer \verb addNode.out .
 
 \subsubsection*{Resultat}
 Correcte.
 \newpage
 
 \section{addEdge}
 \subsubsection*{Descripció}
 Els casos d'ús que es proven són:
 \begin{itemize}
  \item Afegir un node al graf.
  \item Afegir una aresta al graf.
 \end{itemize}
 S'afegeix un node autor i un altre paper i una aresta entre ells.
 
 \subsubsection*{Objectius}
 Afegir arestes.
 
 \subsubsection*{Entrada}
 Al fitxer \verb addEdge.in .
 
 \subsubsection*{Sortida}
 Al fitxer \verb addEdge.out .
 
 \subsubsection*{Resultat}
 Correcte.
 \newpage
 
 \section{addWrongEdge}
 \subsubsection*{Descripció}
 Els casos d'ús que es proven són:
 \begin{itemize}
  \item Afegir una aresta al graf (cas de tipus incompatibles).
 \end{itemize}
 S'afegeix un node autor i un altre term, i s'intenta afegir una aresta entre ells.
 
 \subsubsection*{Objectius}
 Comprovar l'avís d'aresta errònia.
 
 \subsubsection*{Entrada}
 Al fitxer \verb addWrongEdge.in .
 
 \subsubsection*{Sortida}
 Al fitxer \verb addWrongEdge.out .
 
 \subsubsection*{Resultat}
 Correcte.
 
 
 \newpage\section{removeNode}
 \subsubsection*{Descripció}
 Els casos d'ús que es proven són:
 \begin{itemize}
  \item Afegir un node al graf.
  \item Esborrar un node del graf.
 \end{itemize}
 S'afegeix un node i després s'esborra.
 
 \subsubsection*{Objectius}
 Esborrar un node del graf.
 
 \subsubsection*{Entrada}
 Al fitxer \verb removeNode.in .
 
 \subsubsection*{Sortida}
 Al fitxer \verb removeNode.out .
 
 \subsubsection*{Resultat}
 Correcte.
 \newpage
 
 \section{removeEdge}
 \subsubsection*{Descripció}
 Els casos d'ús que es proven són:
 \begin{itemize}
  \item Esborrar una aresta.
 \end{itemize}
 S'afegeixen dos nodes i una aresta entre ells. Després s'esborra l'aresta.
 
 \subsubsection*{Objectius}
 Esborrar arestes.
 
 \subsubsection*{Entrada}
 Al fitxer \verb removeEdge.in .
 
 \subsubsection*{Sortida}
 Al fitxer \verb removeEdge.out .
 
 \subsubsection*{Resultat}
 Correcte.
 \newpage
 
 \section{dupNames}
 \subsubsection*{Descripció}
 Els casos d'ús que es proven són:
 \begin{itemize}
  \item Distingir nodes de nom repetit.
 \end{itemize}
 S'afegeixen dos nodes amb el mateix nom. Després s'intenta esborrar un i surt 
 un avís per triar quin dels dos s'ha d'esborrar.
 
 \subsubsection*{Objectius}
 Distingir nodes de nom repetit.
 
 \subsubsection*{Entrada}
 Al fitxer \verb dupNames.in .
 
 \subsubsection*{Sortida}
 Al fitxer \verb dupNames.out .
 
 \subsubsection*{Resultat}
 Correcte.
 \newpage
 
\section{jpAddRemRem}
 \subsubsection*{Descripció}
 Els casos d'ús que es proven són:
 \begin{itemize}
  \item Afegir un node al graf.
  \item Esborrar un node del graf.
  \item Esborrar un node del graf (cas en que no existeix).
 \end{itemize}
 S'afegeix un autor i s'esborra dues vegades. A la segona s'avisa que no existeix cap node.
 
 \subsubsection*{Objectius}
 Afegir i esborrar nodes correctament, avisant a l'usuari quan no existeix el node introduït.
 
 \subsubsection*{Entrada}
 Al fitxer \verb jpAddRemRem.in .
 
 \subsubsection*{Sortida}
 Al fitxer \verb jpAddRemRem.out .
 
 \subsubsection*{Resultat}
 Correcte.
 \newpage
 
 
 \section{jpEdges}
 \subsubsection*{Descripció}
 Els casos d'ús que es proven són:
 \begin{itemize}
  \item Afegir una aresta.
  \item Consultar nodes i arestes.
  \item Esborrar una aresta.
  \item Esborrar una aresta (cas en que no existeix).
 \end{itemize}
 S'afegeixen dos autors i un paper, i una aresta entre cada autor i el paper. Es consulten els nodes i arestes. Després
 s'esborra l'aresta entre el paper i un dels autors, dues vegades: la primera s'esborra amb èxit, a la segona surt un 
 missatge aclarint que l'aresta no existeix. Es torna a consultar l'estat del graf.
 
 \subsubsection*{Objectius}
 Afegir i esborrar arestes correctament, avisant a l'usuari quan no existeix l'aresta introduida. 
 Consultar nodes i arestes.
 
 \subsubsection*{Entrada}
 Al fitxer \verb jpEdges.in .
 
 \subsubsection*{Sortida}
 Al fitxer \verb jpEdges.out .
 
 \subsubsection*{Resultat}
 Correcte.
 \newpage
 
  
 \section{jpInvalidTypes}
 \subsubsection*{Descripció}
 Els casos d'ús que es proven són:
 \begin{itemize}
  \item Afegir una aresta (entre tipus incompatibles).
  \item Afegir una aresta (entre un node que existeix i un altre que no).
  \item Esborrar un node del graf (cas en que no existeix).
 \end{itemize}
 Afegeix tres nodes, un paper, un term i una conf. Intenta afegir una aresta entre
 la conf i el term i no es pot per tipus incompatibles. Intenta afegir una aresta 
 entre el paper i un autor inexistent, i tampoc es permet. Intenta esborrar un 
 autor inexistent i surt un avís.
 
 \subsubsection*{Objectius}
 Comprovar si surten avisos quan s'intenten afegir arestes impossibles.
 
 \subsubsection*{Entrada}
 Al fitxer \verb jpInvalidTypes.in .
 
 \subsubsection*{Sortida}
 Al fitxer \verb jpInvalidTypes.out .
 
 \subsubsection*{Resultat}
 Correcte.
  \newpage
 
 
  \section{jpNeighAfterRemNode}
 \subsubsection*{Descripció}
 Els casos d'ús que es proven són:
 \begin{itemize}
  \item Esborrar un node (que té arestes).
  \item Consulta de veïns (després d'esborrar algun veï).
 \end{itemize}
 Afegeix un node terme i un altre paper, i una aresta entre ells.
Despres esborra el node terme i consulta els veins del paper. 
El node terme ja no hi apareix.
 
 \subsubsection*{Objectius}
 Comprovar si quan s'esborren nodes, s'esborra correctament tot el seu rastre 
 del graf.
 
 \subsubsection*{Entrada}
 Al fitxer \verb jpNeighAfterRemNode.in .
 
 \subsubsection*{Sortida}
 Al fitxer \verb jpNeighAfterRemNode.out .
 
 \subsubsection*{Resultat}
 Correcte.
  \newpage
  
  \section{query1to1}
 \subsubsection*{Descripció}
 Els casos d'ús que es proven són:
 \begin{itemize}
  \item Consulta 1 a 1 (quan el HeteSim és positiu).
  \item Consulta 1 a 1 (quan el HeteSim és zero).
  \item Consulta 1 a 1 (quan un dels nodes no existeix).
 \end{itemize}
 S'afegeix el graf d'exemple del article de Shi et al. i es prova la consulta 
 1 a 1 en els casos especificats.
 \subsubsection*{Objectius}
 Comprovar la funcionalitat de consulta 1 a 1.
 
 \subsubsection*{Entrada}
 Al fitxer \verb query1to1.in .
 
 \subsubsection*{Sortida}
 Al fitxer \verb query1to1.out .
 
 \subsubsection*{Resultat}
 Correcte.
 \newpage
 
 \section{query1toN}
 \subsubsection*{Descripció}
 Els casos d'ús que es proven són:
 \begin{itemize}
  \item Consulta 1 a N.
 \end{itemize}
 S'afegeix el graf d'exemple del article de Shi et al. i es prova la consulta 
 1 a N.
 \subsubsection*{Objectius}
 Comprovar la funcionalitat de consulta 1 a N.
 
 \subsubsection*{Entrada}
 Al fitxer \verb query1toN.in .
 
 \subsubsection*{Sortida}
 Al fitxer \verb query1toN.out .
 
 \subsubsection*{Resultat}
 Correcte.
 \newpage
 
 \section{queryNtoN}
 \subsubsection*{Descripció}
 Els casos d'ús que es proven són:
 \begin{itemize}
  \item Consulta N a N (quan hi ha tipus intermig).
  \item Consulta N a N (quan no hi ha tipus intermig).
 \end{itemize}
 S'afegeix el graf d'exemple del article de Shi et al. i es prova la consulta 
 N a N en els casos especificats.
 \subsubsection*{Objectius}
 Comprovar la funcionalitat de consulta N a N.
 
 \subsubsection*{Entrada}
 Al fitxer \verb queryNtoN.in .
 
 \subsubsection*{Sortida}
 Al fitxer \verb queryNtoN.out .
 
 \subsubsection*{Resultat}
 Correcte.
 \newpage
 
 \section{queryByReference}
 \subsubsection*{Descripció}
 Els casos d'ús que es proven són:
 \begin{itemize}
  \item Consulta N a N (quan hi ha tipus intermig).
  \item Consulta N a N (quan no hi ha tipus intermig).
 \end{itemize}
 S'afegeix el graf d'exemple del article de Shi et al. i es prova la consulta 
 per referència.
 \subsubsection*{Objectius}
 Comprovar la funcionalitat de consulta per referència.
 
 \subsubsection*{Entrada}
 Al fitxer \verb queryByReference.in .
 
 \subsubsection*{Sortida}
 Al fitxer \verb queryByReference.out .
 
 \subsubsection*{Resultat}
 Correcte.
 \newpage
 
  \section{queryAndResult}
 \subsubsection*{Descripció}
 Els casos d'ús que es proven són:
 \begin{itemize}
  \item Gestió de filtres al resultat.
  \item Ordenació del resultat.
 \end{itemize}
 S'afegeix el graf d'exemple del article de Shi et al. i es prova la consulta 
 N a N. Amb el resultat obtingut s'apliquen els tipus possibles de filtres i 
 s'ordena.
 \subsubsection*{Objectius}
 Comprovar la funcionalitat del resultat.
 
 \subsubsection*{Entrada}
 Al fitxer \verb queryAndResult.in .
 
 \subsubsection*{Sortida}
 Al fitxer \verb queryAndResult.out .
 
 \subsubsection*{Resultat}
 Correcte.
 \newpage
 \end{document}