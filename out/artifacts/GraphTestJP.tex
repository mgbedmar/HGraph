\documentclass[a4paper,11pt]{article}
\usepackage[utf8]{inputenc}
\usepackage[catalan]{babel}
\usepackage{amsmath}
\usepackage{amsthm}
\usepackage{amssymb}
\usepackage{graphicx}
\usepackage[left=3cm,top=3cm,right=2.5cm,bottom=2.5cm]{geometry}
\usepackage{verbatim}
\usepackage{hyperref}

\title{Especificació dels jocs de prova per Graph}
\author{Grup 14.2}
\date{ }

\begin{document}
\maketitle
\tableofcontents
\newpage

\section{jpAddRemRem}
 \subsubsection*{Descripció}
 Els casos d'ús que es proven són:
 \begin{itemize}
  \item Afegir un node al graf.
  \item Esborrar un node del graf.
  \item Esborrar un node del graf (cas en que no existeix).
 \end{itemize}
 S'afegeix un autor i s'esborra dues vegades. A la segona s'avisa que no existeix cap node.
 
 \subsubsection*{Objectius}
 Afegir i esborrar nodes correctament, avisant a l'usuari quan no existeix el node introduït.
 
 \subsubsection*{Entrada}
 Al fitxer \verb jpAddRemRem.in .
 
 \subsubsection*{Sortida}
 Al fitxer \verb jpAddRemRem.out .
 
 \subsubsection*{Resultat}
 Correcte.
 \newpage
 
 
 \section{jpEdges}
 \subsubsection*{Descripció}
 Els casos d'ús que es proven són:
 \begin{itemize}
  \item Afegir una aresta.
  \item Consultar nodes i arestes.
  \item Esborrar una aresta.
  \item Esborrar una aresta (cas en que no existeix).
 \end{itemize}
 S'afegeixen dos autors i un paper, i una aresta entre cada autor i el paper. Es consulten els nodes i arestes. Després
 s'esborra l'aresta entre el paper i un dels autors, dues vegades: la primera s'esborra amb èxit, a la segona surt un 
 missatge aclarint que l'aresta no existeix. Es torna a consultar l'estat del graf.
 
 \subsubsection*{Objectius}
 Afegir i esborrar arestes correctament, avisant a l'usuari quan no existeix l'aresta introduida. 
 Consultar nodes i arestes.
 
 \subsubsection*{Entrada}
 Al fitxer \verb jpEdges.in .
 
 \subsubsection*{Sortida}
 Al fitxer \verb jpEdges.out .
 
 \subsubsection*{Resultat}
 Correcte.
 \newpage
 
  
 \section{jpInvalidTypes}
 \subsubsection*{Descripció}
 Els casos d'ús que es proven són:
 \begin{itemize}
  \item Afegir una aresta (entre tipus incompatibles).
  \item Afegir una aresta (entre un node que existeix i un altre que no).
  \item Esborrar un node del graf (cas en que no existeix).
 \end{itemize}
 Afegeix tres nodes, un paper, un term i una conf. Intenta afegir una aresta entre
 la conf i el term i no es pot per tipus incompatibles. Intenta afegir una aresta 
 entre el paper i un autor inexistent, i tampoc es permet. Intenta esborrar un 
 autor inexistent i surt un avís.
 
 \subsubsection*{Objectius}
 Comprovar si surten avisos quan s'intenten afegir arestes impossibles.
 
 \subsubsection*{Entrada}
 Al fitxer \verb jpInvalidTypes.in .
 
 \subsubsection*{Sortida}
 Al fitxer \verb jpInvalidTypes.out .
 
 \subsubsection*{Resultat}
 Correcte.
  \newpage
 
 
  \section{jpNeighAfterRemNode}
 \subsubsection*{Descripció}
 Els casos d'ús que es proven són:
 \begin{itemize}
  \item Esborrar un node (que té arestes).
  \item Consulta de veïns (després d'esborrar algun veï).
 \end{itemize}
 Afegeix un node terme i un altre paper, i una aresta entre ells.
Despres esborra el node terme i consulta els veins del paper. 
El node terme ja no hi apareix.
 
 \subsubsection*{Objectius}
 Comprovar si quan s'esborren nodes, s'esborra correctament tot el seu rastre 
 del graf.
 
 \subsubsection*{Entrada}
 Al fitxer \verb jpNeighAfterRemNode.in .
 
 \subsubsection*{Sortida}
 Al fitxer \verb jpNeighAfterRemNode.out .
 
 \subsubsection*{Resultat}
 Correcte.
  \newpage
 
\end{document}
